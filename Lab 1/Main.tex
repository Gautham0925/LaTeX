\documentclass[12pt,a4paper]{article}
\usepackage[left=2cm,right=2cm,top=2cm,bottom=2cm]{geometry}
\usepackage{fancyhdr}
\begin{document}
% Set the page style to "fancy"...
\pagestyle{fancy}
\title{Introduction to Latex}
\fancyhf{} % clear existing header/footer entries
% We don't need to specify the O coordinate
\fancyhead{} % clear all header fields
\fancyhead[R]{Basics of Latex}
\fancyfoot{} % clear all footer fields
\fancyfoot[LO,CE]{Alva's Institute of Engineering and Technology}
\fancyfoot[R]{\thepage}
\maketitle
\section{What is Latex?}
Introduction
LaTeX is a document typesetting system
Pronounced either “Lay-tech” or “Lah-tech”
Used to produce high-quality technical/scientific documents
such as articles, books, theses, technical reports etc.
Stable, Platform independent - Windows, Linux, Mac OS
We can concentrate purely on typing the contents of the
document; formatting will be taken care by the LaTeX
Free of cost!
LaTeX is a document preparation system for high-quality typesetting
\section{Why Latex is Importnat?}
Used to produce high-quality technical/scientific documents
such as articles, books, theses, technical reports etc.
\section{What is the Free Software Movement?}
The free software movement campaigns to win for the users of computing the freedom 
that comes from free software. Free software puts its users in control of their own 
computing. Non-free software puts its users under the power of the software's developer. 
\\
\section{What is Free Software?}
\textbf{Free software means the users have the freedom to run, copy, distribute, study, 
change and improve the software.}
Free software is a matter of liberty, not price. To understand the concept, you should 
think of "free" as in "free speech," not as in "free beer". More precisely, free software 
means users of a program have the four essential freedoms:
\begin{itemize}
\item The freedom to run the program as you wish, for any purpose (freedom 0).
\item The freedom to study how the program works, and change it so it does your
computing as you wish (freedom 1). Access to the source code is a precondition for this.
\item The freedom to redistribute copies so you can help others (freedom 2)
\item The freedom to distribute copies of your modified versions to others (freedom 3). 
By doing this you can give the whole community a chance to benefit from your changes. 
Access to the source code is a precondition for this.
Developments in technology and network use have made these freedoms even more 
important now than they were in 1983. Nowadays the free software movement goes far 
beyond developing the GNU system. 
\end{itemize}
\end{document}