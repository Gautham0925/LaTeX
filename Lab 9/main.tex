\documentclass{article}
\usepackage{graphicx} % Required for inserting images
\begin{document}
\begin{center}
 \Large{\textbf{References Demo}}
\end{center}
\section{Introduction}
%\section{Related Work}
For disaster management, uncertainty handling is the main key problem. But, in Joint 
Service deployment and Requests Allocation~(JSR) domain, research work mainly uses the 
approaches such as deterministic optimization \cite{hardtoshare, multicell, bandwidth}, 
Lyapunov optimization \cite{dataintensive}, stochastic optimization, replication of services 
to achieve high reliability, and forecasting of user requests using machine learning without 
considering uncertainty. In deterministic optimization \cite{edgeuav}, request demand is 
known before the run. However, in online optimization, time is divided into slots and 
performs optimization per slot basis, which does not consider uncertain demand. Even if we 
used any probability distribution to model demand, it does not provide the correct 
model/pattern to define the uncertain data \cite{edgeuncertainty}. Using a replication 
approach to achieve high availability also incurs extra resource cost \cite{robust}. Using 
the forecasting method also, we can not predict the impact of uncertain events on the 
requests, which may lead to under-provisioning/over-provisioning resources to process the 
required tasks \cite{rsome}.
\section{Experiment Setup and Performance Parameters}
To demonstrate the efficiency of the proposed approaches, we will simulate the scenario 
for an urban site affected by any natural calamity \cite{oilindustry}. To implement 
optimization models, we will use the IBM Cplex Optimizer tool \cite{cplex}.
\bibliographystyle{IEEEtran}
\bibliography{ref}
\end{document}
